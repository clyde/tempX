%+++Art des Dokuments+++++++++++++++++++++++++++++++++++++++++++++++++++++++++++++++++++++++++++++++
\documentclass[a4paper,12pt, DIV12]{scrartcl} % zuerst Angabe der Papiergr�sse, danach Angabe der Schrift-
% gr�sse in Punkt.
% Die dritte Angabe beeinflusst die Seitenr�nder. Diese k�nnte man mittels 
% \typearea[RandZumBinden]{Breite}{H�he} auch manuell einstellen. Um aber m�glichst gute (typographisch gesehen)
% Seitenlayouts zu erreichen, empfiehlt sich obenstehende Angabe. DIV gefolgt von einer Zahl zwischen 6 und 15,
% wobei 15 extrem kleine R�nder erzeugt.

% Am Ende die Dokumentklassen: scrartcl: f�r kleine bis mittlere Dokumente, 
% scrreprt: f�r grosse Dokumente, scrbook: f�r B�cher

%+++Grundeinstellungen++++++++++++++++++++++++++++++++++++++++++++++++++++++++++++++++++++++++++++++
\usepackage[ngerman]{babel}			%Trennungen, Schriftsatz; Neue deutsche Rechtschreibung
\usepackage[T1]{fontenc}				%Umlaute, Sonderzeichen...				
\usepackage[ansinew]{inputenc}	%Dateicodierung: Unter Linux latin1 anstatt ansinew verwende
																%F�r Betriebssysteme mit utf8-Codierung (einige Unixe):
																%\usepackage{ucs}  \usepackage[utf8]{inputenc}
\usepackage{graphicx}						%Paket um Grafiken einzubinden. Evtl. muss unter Windows  
																% mit \usepackage[dvips]{graphicx} der dvips-Treiber f�r EPS-Grafiken geladen werden
\usepackage{palatino}						%Schriftart - hier k�nnte auch times oder helvet stehen
																%wird zwar von l2tabu nicht empfohlen - finde ich pers�nlich aber
																%die "sch�nste" Varianten
\usepackage{multicol}						%Paket f�r mehrspaltige Dokumente

\usepackage{float}
\usepackage{picins}
\usepackage{longtable}

%+++Glossar+++++++++++++++++++++++++++++++++++++++++++++++++++++++++++++++++++++++++++++++++++++++
\usepackage[nonumberlist]{glossaries}

% Glossar erstellen
\newglossary[slg]{symbols}{sym}{sbl}{List of Symbols}

% alle Begriffe des Glossars
\newglossaryentry{jee}{name=JEE,description={Java Enterprise Edition},first={Java Enterprise Edition (JEE)}}

\makeglossaries

%+++Kopf- und Fusszeilen++++++++++++++++++++++++++++++++++++++++++++++++++++++++++++++++++++++++++
\usepackage{scrpage2}						%An Koma-Script optimierte Kopfzeilenklasse, jedoch auf gut f�r
																%andere Dokumentklassen zu verwenden
																%Mit diesem Paket sind auch Kopf- und Fusszeilen m�glich, die 
																%Unterschiede f�r rechte und linke Seiten machen (bspw f�r B�cher)

%Hier folgen die Kopfzeilentexte
\ihead{INSERT}
\chead{}
\ohead{}
\ifoot{INSERT}
\cfoot{}
\ofoot{\pagemark}
% n�tzlich: \pagemark = Seitenzahl

\setheadsepline{1pt}						%Dicke der Trennlinie Kopfzeile - Text
\setfootsepline{0.5pt}					%Dicke der Trennlinie Fusszeile - Text

\pagestyle{scrheadings}					%gemachte Einstellungen anwenden

%ANMERKUNG: Das Paket scrpage2 hat noch viele weitere Einstellungsm�glichkeiten. Die Dokumentation dazu
%finden sie beispielsweise hier: http://www.ctan.org/tex-archive/macros/latex/contrib/koma-script/scrguide.pdf

%+++Linienabstand+++++++++++++++++++++++++++++++++++++++++++++++++++++++++++++++++++++++++++++++++

%\linespread{1.2}	              %f�r kleine Zeilenabstandskorrekturen.
																%Soll aber der Zeilenabstand in den Fussnoten beibehalten werden,
																%so muss man das Paket setspace.sty verwenden:
\usepackage{setspace}
\onehalfspacing									%1.5 Zeilenabstand; 1 = \singlespacing; 2 = \doublespacing


%+++Absatzeinzug++++++++++++++++++++++++++++++++++++++++++++++++++++++++++++++++++++++++++++++++++
\setlength{\parindent}{1em}			% 1em = Gr�sse, die ein grosses M der aktuellen Schrift
																% Platz braucht: Somit ist diese Gr�sse Schriftabh�ngig
																% (was auch Sinn macht)

%Schusterjungen und Hurenkinder vermeiden 
\clubpenalty=5000\relax 
\widowpenalty=5000\relax 
%Maximalwert ist 10000

%+++Hier beginnt das eigenliche Dokument++++++++++++++++++++++++++++++++++++++++++++++++++++++++++ 
\begin{document}
  %---Titelseite------------------------------------------------------------------------------------
\begin{titlepage} 							%Beginn der Titelseite
\title{INSERT}
\author{INSERT}
\date{Revision: 1.0 - \today}
\maketitle
\end{titlepage}									%Ende der Titelseite \newpage
  \tableofcontents \newpage
  \section{Zielbestimmungen}

\gls{jee}

\subsection{Musskriterien} 
\begin{itemize}
	\item Verwalten von Projekten
	\item Verwalten von Bildmengen
	\item Verwalten von Auswertungen
	\item Exportieren von Auswertungen in JPEG
	\item Auswertungstypen
		\begin{itemize}
			\item Boxplot
			\item 2D Histogramm 
			\item 3D Histogramm 
			\item 3D Cluster
		\end{itemize}
	\item Auszuwertende EXIF Daten
			\begin{itemize}
			\item Kameramodel
			\item Blende 
			\item Verschlusszeit
			\item ISO-Wert
			\item Brennweite
			\item Uhrzeit
			\item Wochentag
			\item Datum
			\item Objektivname, falls vorhanden
		\end{itemize}
	
\end{itemize}

\subsection{Wunschkriterien} 
\begin{itemize}
	\item Normierung von Werten, z.B. Brennweitenkorrektur
	\item Tabelle als Auswertungstyp
\end{itemize}

\subsection{Abgrenzungskriterien} 
\begin{itemize}
	\item \gls{tempX} soll keine EXIF Daten bearbeiten.
\end{itemize}
 \newpage
  \section{Produkteinsatz}

\subsection{Anwendungsbereiche}

\subsection{Zielgruppe}
	\begin{itemize}
		\item Hobby- sowieso Freizeitfotografen
		\item Profifotografen		
	\end{itemize}

\subsection{Betriebbedingungen}
 \newpage
  \section{Produktumgebung}

Das Programm l�uft auf einem der Poolrechner im Raum 356 des Informatik Geb�udes 50.34 des Karlsruher Institut of Technologies.

\begin{itemize}
\item Software
	\begin{itemize}
	\item Betriebssystem: 
		\begin{itemize}
			\item Windows 2000/XP/Vista/7
			\item Linux
			\item (optional) Mac OS X 10.5
		\end{itemize}
	
	\item Laufzeitumgebung:
		\begin{itemize}
			\item Java 1.6
		\end{itemize}		
	\end{itemize}
	
\item Hardware 
	\begin{itemize}
	\item Mindestanforderung an den Arbeitsplatzrechner: 
		\begin{itemize}
			\item Dual Core 2 Ghz
			\item 2 GB RAM
			\item Bildschirm mit einer Aufl�sung von 800 x 600 Pixel
			\item Hardbwarebeschleunigte 3D Unterst�tzung
		\end{itemize}
	
	\item Kamera:
		\begin{itemize}
			\item Alle Kameramodelle die mindestens den JEITA Exif Version 2.1 Standard vom 1. Juni 1998 einhalten
		\end{itemize}		
	\end{itemize}	
\end{itemize} \newpage
  \section{Funktionale Anforderungen}

\begin{itemize}
	\item /F10/\\ Bereits erstellte Auswertungen k�nnen als Vorlage f�r neue Auswertungen verwendet werden.
\end{itemize} \newpage
  \section{Produktdaten} \newpage
  \section{Nichtfunktionale Anforderungen}

\begin{itemize}
	\item /NF10/\\Das Einlesen und extrahieren der Exif-Daten sollte pro 1.000 Bildern maximal 2 Minuten und 30 Sekunden brauchen.
	\item /NF20/\\Ein Projekt muss mit einer Bildmenge von 10.000 Bildern umgehen k�nnen, ohne dass ein Programmabsturz oder l�ngerfristigen Programmunterbrechungen daraus resultieren.
	\item /NF30/\\Bedienfehler d�rfen nicht dazu f�hren, dass Daten verloren gehen.
	\item /NF40/\\Die grafische Benutzerschnittstelle sollte so gestaltet sein, dass ein unerfahrener Benutzer sich in angemessener Zeit einarbeiten kann.
	\item /NF50/\\Das Produkt enth�lt nicht mehr als 1\% plattformspezifischer Anweisungen.
\end{itemize} \newpage
  \section{Globale Testf�lle} \newpage
  \section{Systemmodelle} \newpage
  \subsection{Szenarien} \newpage
  \subsection{Anwendungsf�lle} \newpage
  \subsection{Objektmodell} \newpage
  \subsection{Dynamische Modelle} \newpage
  \subsection{Benutzerschnittstelle} \newpage
  %+++Workaround++++++++++++++++++++++++++++++++++++++++++++++++++++++++++++++++++++++++++++++++++++

% Durch diesen Workaround, wird das Glossar in der TOC und auf der Seite richtig nummeriert
\stepcounter{section}
\addcontentsline{toc}{section}{\numberline {\thesection} Glossar}

%+++Ausgabe+++++++++++++++++++++++++++++++++++++++++++++++++++++++++++++++++++++++++++++++++++++++

% Schreibt das Glossar
\printglossary[style=list,title=\thesection~Glossar]
\end{document}
