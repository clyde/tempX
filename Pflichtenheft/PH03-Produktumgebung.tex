\section{Produktumgebung}

\gls{tempX} l�uft auf einem der Poolrechner im Raum 356 des Informatikbaus (Geb 50.34) des \gls{kit}.

\subsection{Software}

	\begin{itemize}
		
		\item Betriebssystem: 
		\begin{itemize}
			\item Windows XP/Vista/7
			\item Linux
			\item (optional) Mac OS X 10.6
		\end{itemize}
	
		\item Laufzeitumgebung:
		\begin{itemize}
			\item Java 1.6
		\end{itemize}
		
	\end{itemize}
	
\subsection{Hardware}

	\begin{itemize}
		
		\item Mindestanforderung an den Arbeitsplatzrechner: 
		\begin{itemize}
			\item Dual Core 2 Ghz
			\item 2 GB RAM
			\item Bildschirm mit einer Aufl�sung von 720 x 500 Pixel
			\item 20 MB freier Speicherplatz auf der Festplatte
		\end{itemize}
	
		\item empfohlene Anforderungen an den Arbeitsplatzrechner:
		\begin{itemize}
			\item Intel\textregistered Core\texttrademark 2 Quad Q6600 2,4 Ghz
			\item 8 GB RAM
		\end{itemize}	
	
		\item Kamera:
		\begin{itemize}
			\item Alle Kameramodelle, die mindestens den JEITA Exif Version 2.1 Standard vom 1. Juni 1998 einhalten
		\end{itemize}
		
	\end{itemize}